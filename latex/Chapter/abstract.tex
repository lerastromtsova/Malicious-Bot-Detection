% !TeX spellcheck = en_US
% !TeX encoding = UTF-8
\chapter*{Abstract}
This thesis concerns automatic detection of malicious political bots in Russian social networks on the example of VKontakte discussions around the Russian-Ukrainian armed conflict of 2022. This study attempts to close several gaps in the existing research around social bots. Firstly, it is one of the few studies investigating bot activity on an example of content produced in languages other than English. Secondly, it explores the effect political bots have internally in Russian-speaking communities and not the influence they produce on Western society. Thirdly, it presents a bot detection web tool available to the public that has no alternatives in the Russian-speaking segment of the Internet. 

The thesis aims to build a model capable of detecting malicious political bots on VKontakte and explore the influence these bots produce on the online environment. The model is applied to VKontakte data in order to uncover political manipulations and capture the influence that bots have on this platform. We explore bots' influence in terms of the influentialness and sentiments of the user network. The results of the model can benefit both society and business because they all profit from increased transparency and level of trust in a social network as a result of the timely removal of bots and bot-produced content.  We make the model results accessible to a broad audience on the Internet by building a web application "Bot-Checker". This allows anyone to see whether a VKontakte user is a bot or a real human. 

The accuracy of the bot detection model, estimated on a subset of the data, reaches 85\%. However, finding ways to confidently identify ground truth for bot-human classification and build other reliable bot detection models remains a crucial task for future research endeavours.
