\chapter{Conclusion and Future Work}
\label{ch:conclusion}

\section{Future work}
Opportunities for future work arise naturally from the limitations of the current study. To advance the detection of political bots in Russian-language social networks, such as VKontakte, it is necessary to address these limitations and solve them.

Firstly, a promising research area is the development of sentiment analysis techniques for less widespread languages than English and Russian. For instance, in this study, a model for sentiment analysis of Ukrainian texts would be beneficial. 

Secondly, a challenging task for future research arises from the absence of ground truth to train and evaluate the models. The scientific community has long known that the current evaluation and training techniques are not perfect. However, only one reliable method of obtaining the ground truth has been invented. This method is not applicable to all studies and was not used in this research. Thus, solving the problem of absence of ground truth might benefit many researchers.

Thirdly, it is possible to try and build the users' graph based on a different behavioural trace than the friendship connection, URL sharing or hashtag sequences. This can produce results varying from those that emerged during this study. For example, users' similarity is one of the traces that could be researched, although it would require significant computational resources to be applied to large user networks.

Lastly, improvements to the web bot detection tool present a potential direction for future scientific endeavours. Extending the database with more users, analysing new clusters, making more information available in the web interface and improving the system based on users' feedback might be beneficial for the end users and help to raise awareness of the presence of bots in the online social environment.

\section{Conclusion}
Identifying political bots in online social networks is crucial for society and businesses due to their omnipresence and significant influence on the online environment. By influencing the online environment, bots can manipulate OSN users' opinions and help malicious individuals spread their agenda according to their goals. This thesis concerns detecting and investigating the behaviour of VKontakte political bots related to the Russian-Ukrainian armed conflict of 2022. 

This research offers several contributions to political bot detection on Russian social networks. 

Firstly, it presents one of the first attempts to build a bot detection model suitable for the identification of political bots on VKontakte, a major Russian OSN. The model is graph- and group-based, unsupervised, and relies on clustering to uncover coordinated communities of users. 

Secondly, this research provides insights into political bot behaviour in the VKontakte discourse around the Russian-Ukrainian armed conflict of 2022. Judging by the centrality characteristics of the user network, bots seem to be influential actors in the online community. Moreover, bots influence the sentiments expressed in the network by producing more emotional content than humans and primarily amplifying negative emotions.

Thirdly, as a result of this research, a web application was developed to allow users to check if a user on VKontakte is a bot or a real person. The web application might be a tool to increase the transparency of social networks and increase awareness of political bots in the online environment.

Future research on this topic can take several directions in order to extend this study and improve its results. Working with the limitations of this research, especially concerning the model evaluation and finding the ground truth, may present significant interest to the social bot researchers interested in exploring the specifics of the modern informational war on social networks. In this emerging area of study, hopefully, this paper will become a foundation for future endeavours to uncover political manipulations, educate social network users and help people form unbiased and objective opinions without the influence of malicious social bots.
